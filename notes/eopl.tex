\documentclass{article}
\usepackage{amsmath}
\usepackage{amsthm}
\usepackage{amsfonts}
%\usepackage{parskip}
\newtheorem{name}{Printed output}
\newtheorem{mydef}{Definition}
\begin{document}

\section*{Section 1.1}

We motivate the following discussion by noting that the syntax of a program 
in a language is usually a nested or tree like structure. Recursion is then 
an important technique in constructing and manipulating such structures.  

Inductive specification is a method for specifying a set of values. For example,
let's consider a set $ S \subseteq \mathbb{N} $. We can define $S$ as follows: 

\begin{mydef} \textit{A natural number} $n \in S$ \textit{if and only if}  
\begin{enumerate} 
  \item $n = 0$, or 
  \item $ n - 3 \in S $. 
\end{enumerate}
\textit{We call this definition the \textbf{top down definition} of $S$}
\end{mydef} 


\begin{mydef} \textit{Define the set $S$ to be the smallest set contained in $\mathbb{N}$ and satisfying the following two properties:} 
\begin{enumerate}
  \item $0 \in S$, and 
  \item if $n \in S$, then $(n + 3) \in S$. 
\end{enumerate}
\textit{We call this definition the \textbf{bottom up definition} of $S$}
\end{mydef}


\begin{mydef} $$ \frac{}{0 \in S}$$ $$ \frac{n \in S}{(n + 3) \in S} $$ 
\textit{We call this definition the \textbf{rules of inference} of $S$}
\end{mydef}

\end{document}
