\documentclass{article}
\usepackage{amsmath}
\usepackage{amsthm}
\usepackage{amsfonts}
%\usepackage{parskip}
\newtheorem{name}{Printed output}
\newtheorem{mydef}{Definition}
\begin{document}

\section*{Section 1.1}
Inductive specification is a method for specifying a set of values. For example, let's consider a set $ S \subseteq \mathbb{N} $. We can define $S$ as follows: 
\begin{mydef} \textit{A natural number} $n \in S$ \textit{if and only if}  
\begin{enumerate} 
  \item $n = 0$, or 
  \item $ n - 3 \in S $. 
\end{enumerate}
\end{mydef} 
We can write this definition another way: 
\begin{mydef} \textit{Define the set $S$ to be the smallest set contained in $\mathbb{N}$ and satisfying the following two properties:} 
\begin{enumerate}
  \item $0 \in S$, and 
  \item if $n \in S$, then $(n + 3) \in S$. 
\end{enumerate}
\end{mydef}
And also...
\begin{mydef} $$ \frac{}{0 \in S}$$ $$ \frac{n \in S}{(n + 3) \in S} $$ 
\end{mydef}

The first definition is called the top down definition of $S$. We call it such because 

\end{document}
